\documentclass{article}

% Language setting
% Replace `english' with e.g. `spanish' to change the document language
\usepackage[english]{babel}
\usepackage[scr=boondox,cal=esstix]{mathalpha}
% Set page size and margins
% Replace `letterpaper' with `a4paper' for UK/EU standard size
\usepackage[letterpaper,top=2cm,bottom=2cm,left=3cm,right=3cm,marginparwidth=1.75cm]{geometry}
\usepackage{amsthm}

\newtheorem{theorem}{Theorem}[section]
\newtheorem{lemma}[theorem]{Lemma}

% Useful packages
\usepackage{amsmath}
\usepackage{amssymb}
\usepackage{graphicx}
\usepackage[colorlinks=true, allcolors=blue]{hyperref}

\title{Hartshorne Algebraic Geometry Solution}
\author{Functor}

\begin{document}
\maketitle

\begin{abstract}
This is a collection of solutions to the exercises of Hartshorne's Algebraic Geometry \cite{Hartshorne1977AlgebraicG}. There are already many good solutions like Joe Cutrone and Nick Marshburn's, so this file should be regarded as a complement. That is, I only write solutions to the exercises that Joe Cutrone and Nick Marshburn's do not contain, or the solutions that are different from theirs.

EGA always refer to \cite{EGA}.
\end{abstract}

\section{Varieties}

\section{Schemes}

\section{Cohomology}
\textbf{Exercise 4.11} 
\begin{proof}
This can be proved imitating the proof of (4.5). For convenience, equip $X$ with a constant sheaf $O_X = \mathbb{Z}$ so $(X, O_X)$ is a ringed space. Recall that abelian groups are $\mathbb{Z}$-modules.

(2.2) implies that $\mathscr{F}$ can be embedded into an injective sheaf $\mathscr{G}$. Denote the cokernel by $\mathscr{R}$, then we have an exact sequence of sheaves
\[
0 \longrightarrow \mathscr{F} \longrightarrow \mathscr{G} \longrightarrow \mathscr{R} \longrightarrow 0
\]
Since for open sets like $V = U_{i_0} \cap \dots \cap  U_{i_p}$, $H^1(V, \mathscr{F}|_V) = 0$, we deduce that
\[
0 \longrightarrow \mathscr{F}(V) \longrightarrow \mathscr{G}(V) \longrightarrow \mathscr{R}(V) \longrightarrow 0
\]
is exact. Taking products, we find the $\check{\mathrm{C}}$ech complexes
\[
0 \longrightarrow C^.(\mathcal{U}, \mathscr{F}) \longrightarrow C^.(\mathcal{U}, \mathscr{G}) \longrightarrow C^.(\mathcal{U}, \mathscr{R}) \longrightarrow 0
\]
is exact. By (2.4), $\mathscr{G}$ is flasque. So (4.3) implies its $\check{\mathrm{C}}$ech cohomology vanishes for $p > 0$. So we have an exact sequence
\[
0 \longrightarrow \check{H}^0(\mathcal{U}, \mathscr{F}) \longrightarrow \check{H}^0(\mathcal{U}, \mathscr{G}) \longrightarrow \check{H}^0(\mathcal{U}, \mathscr{R}) \longrightarrow \check{H}^1(\mathcal{U}, \mathscr{F})  \longrightarrow 0
\]
and isomorphisms
\[
\check{H}^p(\mathcal{U}, \mathscr{R}) \xrightarrow{\sim} \check{H}^{p+1}(\mathcal{U}, \mathscr{F})
\]
for $p \geq 1$. Using (2.5) and (4.1), we conclude
\[
\check{H}^1(\mathcal{U}, \mathscr{F}) \xrightarrow{\sim} H^{1}(X, \mathscr{F})
\]
is an isomorphism.

The long exact sequence for cohomology impliest that $\mathscr{R}$ satisfies $\forall k > 0, H^k(V, \mathscr{R}|_V) = 0$, like $\mathscr{F}$. By induction on $p$, $\check{H}^p(\mathcal{U}, \mathscr{F}) \xrightarrow{\sim} H^{p}(X, \mathscr{F})$ is an isomorphism for all $p > 0$.

Obviously, (4.5) is a corollary of this result.
\end{proof}


\bibliographystyle{alpha}
\bibliography{sample}

\end{document}
